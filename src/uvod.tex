\section{Funkcija}

\begin{definition}[funkcija]
    \textbf{Funkcija} $f: X \to Y$ je izraz ili pravilo kojim se svakom elementu
    skupa X (\textit{domene}) pridružuje točno 1 element skupa Y
    (\textit{kodomene}).

    $$
        \forall x \in X,\, \exists! y \in Y \text{ takav da je } y = f(x)
    $$
\end{definition}

\begin{definition}[injecija]
    Funkcija $f$ je \textbf{injecija} ako svakom elementu iz kodomene pridružuje
    \textit{najviše} jedan element iz domene.

    $$
        \forall x,\, x' \in X,\, f(x) = f(x') \implies x = x'
    $$
\end{definition}

\begin{definition}[surjecija]
    Funkcija $f$ je \textbf{surjecija} ako svakom elementu kodomene pridružuje
    \textit{najmanje} jedan element iz domene.

    $$
        \forall y \in Y,\, \exists x \in X \text{ takav da je } y = f(x)
    $$
\end{definition}

\begin{definition}[bijekcija]
    Funkcija $f$ je \textbf{bijekcija} ako svakom elementu iz kodomene pridružuje
    \textit{točno} jedan element iz domene.

    $$
        \forall y \in Y,\, \exists! x \in X \text{ takav da je } y = f(x)
    $$
\end{definition}

\section{Matematička logika}

\begin{definition}[sud]
    \textbf{Sud} (ili \textbf{izjava}) je smislena tvrdnja koja je ili istinita ili lažna.
\end{definition}

\subsection{Operacije nad sudovima}

\begin{tabular}{c|lr}
    $\neg$ & negacija &\\
    $\wedge$ & konjunkcija &(logičko i) \\
    $\vee$ & disjunkcija &(logičko ili) \\
    $\veebar$ & ekskluzivna disjunkcija &(isključivo ili) \\
    $\implies$ & implikacija &\\
    $\Leftrightarrow$ & ekvivalencija &\\
\end{tabular}

\section{Skupovi}

\begin{definition}[skup]
    \textbf{Skup}\index{Skup} je grupacija $n$ međusobno različitih elemenata.
    
    Skup može biti konačan ili beskonačan.
    
    Skup može sadržavati brojeve, druge skupove, relacije, simbole...
\end{definition}

Skup\index{Skup} može biti zadan...
\begin{itemize}
    \item Elementima:
        $$S = \{1, 2, 3, 4, 5\},$$
        $$T = \{a_1, a_2, a_3, a_4, a_5\}.$$
    \item Pomoću pravila:
        $$S = \{n: n \in N \land n > 100\}.$$
\end{itemize}

Duljinu skupa označavamo vertikalnim linijama: $|S|$

\subsection{Učestali skupovi}

\begin{table}[ht]
    \center
    \begin{tabular}{l c p{5cm}}
    \hline
    Skup & & Opis \\ \hline \hline

    Prirodni brojevi\index{Prirodni brojevi} &
    $
    \begin{array}{c}
        \mathbb{N} = \{1, 2, 3, 4, 5, \dots\} \\
        \mathbb{N}_0 = \{0, 1, 2, 3, 4, 5, \dots\}
    \end{array}
    $ &
    Cijeli brojevi veći od nule (i nula, ovisno o literaturi). \\ \hline

    Cijeli brojevi &
    $
    \mathbb{Z} = \{..., -2, -1, 0, 1, 2, ...\}
    $ &
    Unija prirodnih brojeva, njima suprotnih brojeva i nule. \\ \hline

    Racionalni brojevi &
    $\mathbb{Q}$ &
    Svi brojevi koji se mogu dobiti dijeljenjem jednog cijelog broja drugim. \\ \hline

    Iracionalni brojevi & &
    Svi brojevi koji nisu racionalni \\ \hline

    Algebarski brojevi &
    $\mathbb{A}$ &
    Svaki broj koji je rješenje nekog polinoma s racionalnim koeficijentima. \\ \hline

    Transcendentalni brojevi & &
    Brojevi koji nisu algebarski \\ \hline

    Realni brojevi &
    $\mathbb{R}$ &
    Svi racionalni i iracionalni brojevi. \\ \hline

    Imaginarni brojevi &
    $
    \mathbb{I} = \{x: x < 0\}
    $ &
    Brojevi koji daju negativan rezultat pri kvadriranju. \\ \hline

    Kompleksni brojevi &
    $
    \mathbb{C} = \{r + j: r \in \mathbb{R}, j \in \mathbb{I}\}
    $ &
    Sve kombinacije zbrojeva bilo kojeg realnog broja sa bilo kojim imaginarnim brojem. \\ \hline
    \end{tabular}
    \caption{\label{tab:table-name} Skupovi brojeva.}
\end{table}
\newpage

\subsection{Operacije nad skupovima}

\subsubsection{Unija skupova}

Unija je operacija koja vraća skup koji sadrži sve članove operanada. 

Unija je komutativna.

$$
S_1 \cup S_2 = \{x: x \in S_1 \lor x \in S_2\}
$$

Primjeri:

\begin{gather*}
    \{1, 2, 3\} \cup \{2, 3, 4\} = \{1, 2, 3, 4\} \\
    \{3, 4\} \cup \{3, 4\} = \{3, 4\} \\
    \{2, 7\} \cup \{2, 5\} \cup \{10\} = \{2, 5, 7, 10\} \\
    \{\{1, 8\}, \{3, 5\}\} \cup \{\emptyset, \{1, 8\}, \{6, 9\}\} = \{\emptyset,\{1, 8\}, \{3, 5\}, \{6, 9\}\} \\
\end{gather*}

\subsubsection{Presjek skupova}

Presjek je operacija koja vraća skup koji sadrži isključivo članove sadržane u svim operandima. 

Presjek je komutativan.

$$
S_1 \cap S_2 = \{x: x \in S_1 \land x \in S_2\}
$$

Primjeri:

\begin{gather*}
    \{1, 2, 3\} \cap \{2, 3, 4\} = \{2, 3\} \\
    \{3, 4\} \cap \{3, 4\} = \{3, 4\} \\
    \{2, 7\} \cap \{2, 5\} \cap \{2\} = \{2\} \\
    \{\{1, 8\}, \{3, 5\}\} \cap \{\emptyset, \{1, 8\}, \{6, 9\}\} = \{\{1, 8\}\} \\
\end{gather*}

\subsubsection{Razlika skupova}

Razlika nije komutativna.

$$
S_1 \backslash S_2 = \{x: x \in S_1 \land x \notin S_2\}
$$

Primjeri:

\begin{gather*}
    \{1, 2, 3\} \backslash \{2, 3, 4\} = \{1\} \\
    \{3, 4\} \backslash \{3, 4\} = \emptyset \\
    \{2, 7\} \backslash \{2, 5\} \backslash \{10\} = \{7\} \\
    \{\{1, 8\}, \{3, 5\}\} \backslash \{\{\}, \{1, 8\}, \{6, 9\}\} = \{\{3, 5\}\} \\
\end{gather*}

\subsubsection{Kartezijev produkt skupova}

Kartezijev produkt je komutativan.

$$
S_1 \times S_2 = \{(a, b): a \in S_1 \wedge b \in S_2\}
$$

Primjeri:

\begin{gather*}
    \{1, 2\} \times \{3, 4\} = \{(1, 3), (1, 4), (2, 3), (3, 4)\} \\
    \emptyset \times \{1, 2\} = \emptyset \\
    \{\emptyset\} \times \{1, 2\} = \{(\emptyset, 1), (\emptyset, 2)\} \\
\end{gather*}

\subsubsection{Simetrična razlika skupova}

Simetrična razlika skupova je unija svih skupova bez njihovih presjeka.

$$
S_1 \triangle S_2 = (S_1 \backslash S_2) \cup (S_2 \backslash S_1)
$$

\subsubsection{Komplement skupa}

Komplement skupa sadrži sve elemente univerzalnog skupa koji nisu sadržani u komplementiranom skupu.

$$
\overline{S} = \{x \in U: x \notin S\}
$$

\subsubsection{Partitivni skup}

Parititivni skup $\mathbf{P}(S)$ skupa $S$ je skup svih podskupova skupa $S$.

Primjer:
\begin{gather*}
    S = \{1,2,3\}\\
    \mathrm{P}(S) = \{\emptyset, \{1\}, \{2\}, \{3\}, \{1,2\}, \{1,3\}, \{2,3\}, \{1,2,3\}\}
\end{gather*}

\section{Veza između logike, algebre i skupova}

\begin{center}
    \begin{multicols}{2}
        \begin{tabular}{c|c|c}
            Logika & Algebra & Skupovi\\
            \hline
            $\top$ & $1$ & $\mathcal{U}$\\
            $\bot$ & $0$ & $\emptyset$\\
            $\mathrm{I}$ (identitet) & $1$ za $\times$, $0$ za $+$ & $\mathcal{U},\,\emptyset$\\
            $=$ & $=$ & $=$\\

        \end{tabular}

        \columnbreak

        \begin{tabular}{c|c|c}
            Logika & Algebra & Skupovi\\
            \hline
            $\wedge$ & $\times$ & $\cap$\\
            $\vee$ & $+$ & $\cup$\\
            $\oplus$ & $-$ & $\Delta$\\
            $\neg$ & $-x$ & $\overline{A}$\\
            $\implies$ & $\leq$ & $\subseteq$\\
        \end{tabular}
    \end{multicols}
\end{center}
