Rekurzija se pojavljuje kada definicija koncepta ili procesa ovisi o
jednostavnijoj ili prethodnoj verziji sebe.

U kontekstu matematike, to znači da se funkcija pojavljuje unutar svoje
definicije.

\section{Rekurzivne relacije}

Primjeri rekurzivnih relacija su kule hanoja i fibbonacijevi brojevi.

Fibbonacijevi brojevi:
$$
    f_n = \begin{cases}
        f_{n-1} + f_{n-2},& n > 1\\
        1,& n \in [0, 1]
    \end{cases}
$$

\section{Linearne nehomogene rekurzije s konstantnim koeficijentima}

Linearna nehomogena rekurzija s konstantnim koeficijentima reda $r$

$$
a_n = c_1a_{n-1}+c_2a_{n-2}+\dots+c_ra_{n-r}+f(n),\, c_r \neq 0,\, n \geq r,\, f: \mathbb{N} \to \mathbb{C}
$$

Početni uvjeti:
$$
a_0,\, a_1,\, \dots,\, a_{r-1}
$$

Karakteristična jednadžba pripadne homogene rekurzije:
$$
x^r = c_1x^{r-1}+c_2x^{r-2}+\dots+c_r
$$

Karakteristički korijeni pripadne homogene rekurzije:
$$
x_1,x_2,\dots,x_r
$$

Opće rješenje pripadne homogene rekurzije:
$$
a_n^* = \lambda_1x_1^n+\lambda_2x_2^n+\dots+\lambda_rx_r^n
$$

Partikularno rješenje $a'_n$ nehomogene rekurzije ovisi o funkciji $f(n)$.

Opće rješenje rekurzije: $\displaystyle a_n = a_n^*+a'_n$

\begin{center}
    \begin{tabular}{c|c}
        $f(n)$ & $a'_n$ \\
        \hline
        $C$ & $A$ \\
        $Cn$ & $An+B$ \\
        $P_k(n)$ & $Q_k(n)$ \\
        $Cb^n$ & $Ab^n$ \\
        $n^pb^n$ & $b^nQ_p(n)$
    \end{tabular}
\end{center}

\begin{problem}
    Odredite rekurzivnu relaciju za rješenje problema Kule Hanoja.
\end{problem}

\begin{problem}
    Riješite rekurziju $a_n = 2a_{n-1}+2^n,\, a_0 = 5$.
\end{problem}

Ako je $f(n) = a^n$ i ako je $a$ karakterističan korijen kratnosti $k$, onda se pripadno partikularno rješenje množi s $n^k$.

\section{Linearne homogene rekurzije s konstantnim koeficijentima}

Linearna homogena rekurzija s konstantnim koeficijentima reda $r$

$$
    a_n = c_1a_{n-1}+c_2a_{n-2}+\dots+c_ra_{n-r},\, c_r \neq 0,\, n \geq r
$$

Početni uvjeti:
$$
a_0,\, a_1,\, \dots,\, a_{r-1}
$$

Karakteristična jednadžba:
$$
x^r = c_1x^{r-1}+c_2x^{r-2}+\dots+c_r
$$

Karakteristički korijeni:
$$
x_1,x_2,\dots,x_r
$$

\begin{theorem}
    Neka su karakteristički korijeni $x_1,x_2,\dots,x_r$ rekurzivne relacije
    $$
        a_n = c_1a_{n-1}+c_2a_{n-2}+\dots+c_ra_{n-r},\, c_r \neq 0,\, n \geq r
    $$
    međusobno različiti. Tada je opće rješenje rekurzije
    $$
        a_n = \lambda_1x_1^n+\lambda_2x_2^n+\dots+\lambda_rx_r^n
    $$
\end{theorem}

\begin{problem}
    Odredite opći član Fibbonacijevog niza.
\end{problem}

\begin{theorem}
    Neka su $x_1,x_2,\dots,x_r$ različiti korijeni rekurzivne relacije
    $$
        a_n = c_1a_{n-1}+c_2a_{n-2}+\dots+c_ra_{n-r},\, c_r \neq 0,\, n \geq r\,,
    $$
    te neka je $x_i$ korijen kratnosti $v_i, i \in [1, t]$. Tada je partikularno rješenje rekurzije koje odgovara korijenu $x_i, i \in [1, t]$,
    $$
        a_n^{(i)} = \lambda_1^{(i)}x_i^n+\lambda_2^{(i)}nx_i^n+\dots+\lambda_{v_i}^{(i)}n^{v_i-1}x_i^n\,,
    $$
    a opće rješenje rekurzije je
    $$
        a_n = a_n^{(1)}+a_n^{(2)}+\dots+a_n^{(t)}\,.
    $$
\end{theorem}

\section{Linearni sustavi rekurzija}

\begin{example}
    Neka su $(a_n)_{n\in\mathbb{N}_0}$ i $(b_n)_{n\in\mathbb{N}_0}$ nizovi za koje vrijedi:

    \begin{gather*}
        a_{n+1} = c_1a_n+c_2b_n+f_1(n)\,,\\
        b_{n+1} = c_3a_n+c_4b_n+f_2(n)\,,\\
        a_0=c_5,\quad b_0=c_6\,.
    \end{gather*}
\end{example}

Sustav se metodom eliminacije svodi na linearnu rekurziju s konstantnim
koeficijentima.

\section{Racionalne rekurzije}

$$
a_{n+1} = \frac{pa_n+q}{ra_n+s},\quad a_0 = a\,.
$$

Postupak rješavanja:

Neka su $(b_n)_{n\in\mathbb{N}_0}$ i $(c_n)_{n\in\mathbb{N}_0}$ nizovi za koje vrijedi
\begin{gather*}
    b_{n+1} = pb_n + qc_n\\
    c_{n+1} = rb_n + sc_n\\
    b_0=a,\quad c_0 = 1\,.
\end{gather*}

Tada je
$$
    a_{n+1} = \frac{b_{n+1}}{c_{n+1}} = \frac{pa_n+q}{ra_n+s}\,.
$$
