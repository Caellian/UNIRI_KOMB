\section{Binomni koeficijent}

Izražava na koliko se načina može odabrati k od n odabira bez ponavljanja, tj.
broj $k$-kombinacija $n$-članog skupa.

$$
{{n}\choose{k}} = \frac{n!}{(n-k)!k!}
$$

\subsection{Pascalov trokut}

Pascalov trokut predstavlja rješenja za binomne koeficijente. Svaki broj u
trokutu je zbroj dva broja iznad njega.

\begin{tabular}{rccccccccc}
    $n=0$:&    &    &    &    &  1\\\noalign{\smallskip\smallskip}
    $n=1$:&    &    &    &  1 &    &  1\\\noalign{\smallskip\smallskip}
    $n=2$:&    &    &  1 &    &  2 &    &  1\\\noalign{\smallskip\smallskip}
    $n=3$:&    &  1 &    &  3 &    &  3 &    &  1\\\noalign{\smallskip\smallskip}
    $n=4$:&  1 &    &  4 &    &  6 &    &  4 &    &  1\\\noalign{\smallskip\smallskip}
\end{tabular}

Vrijednost $n$-tog reda (ranga) i $k$-tog stupca je jednaka $\binom{n}{k}$. S
tom notacijom je konstrukciju moguće zapisati i kao:

$$
\binom{n}{k} = \binom{n-1}{k-1} + \binom{n-1}{k}
$$

\subsection{Binomni teorem}

Binomni teorem opisuje algebarsko proširenje eksponenta binoma. Prema teoremu,
moguće je proširiti polinom oblika $(x+y)^n$ u zboj izraza oblika $ax^by^c$,
gdje su eksponenti $b$ i $c$ pozitivni cijeli brojevi za koje vrijedi $b+c=n$, a
koeficijent $a$ svakog izraza je određeni pozivni cijeli broj sukladan
odgovarajućem stupcu u pascalovom trokutu

Na primjer, za $n=4$:

$$
(x+y)^4=x^4+4x^3y+6x^2y^2+4xy^3+y^4
$$

U primjeru možemo uočiti da koeficijent $a$ svakog člana prati vrijednosti
stupca na pascalovom trokutu u rangu jednakom eksponentu ($n=4$).

\begin{theorem}[binomni teorem]
    Neka je $n$ prirodan broj i neka su $x$ i $y$ kompleksni brojevi. Tada je

    \begin{align*}
        (x+y)^n &= x^n + \binom{n}{1}x^{n-1}y+\dots+\binom{n}{n-1}xy^{n-1}+y^n\\
        &=\sum_{k=0}^{n} \binom{n}{k} x^{n-k} y^k
    \end{align*}
\end{theorem}

\begin{example}
    Dokažite da je $\displaystyle\sum_{i=0}^{n} \binom{n}{i} = 2^n$.
\end{example}

\subsection{Vandermondova konvolucija}
\textbf{Vandermondova konvolucija} se često koristi za kombinatorne dokaze i rješavanje
sustava jednadžbi u linearnoj algebri.

Primjenjuje se za brojanje kombinacija različitih objekata. Izraz
$$
    \binom{m+n}{r} = \sum_{k=0}^{r} \binom{n}{k}\binom{m}{r-k}
$$
tvrdi da bilo koja od $k$ kombinacija objekata iz grube od $m+n$ objekata mora
imati nekih $0 \leq k \leq r$ objekata iz grupe od $m$ objekata i preostalih $r-k$
objekata iz grube od $n$ objekata.

\begin{theorem}[vandermondova konvolucija]
    \center
    \begin{tabular}{rll}
        $\displaystyle\binom{m}{r}+\binom{n}{1}\binom{m}{r-1}+\dots+\binom{n}{r}$ &= $\displaystyle\sum_{k=0}^{r} \binom{n}{k}\binom{m}{r-k}$ &= $\displaystyle\binom{m+n}{r}$\\
        $\displaystyle\binom{n}{0}\binom{m}{0}+\binom{n}{1}\binom{m}{1}+\dots+\binom{n}{n}\binom{m}{n}$ &= $\displaystyle\sum_{k=0}^{n} \binom{n}{k}\enspace\,\binom{m}{k}$ &= $\displaystyle\binom{m+n}{n}$
    \end{tabular}
\end{theorem}

Iz vandermondove konvolucije također dobivamo i korolar koji govori da je broj
načina na koji možemo odabrati $n$ objekata iz dvije grupe od $n$ objekata
jednak zbroju kvadrata načina na koje možemo odabrati $0 \geq k \geq n$ objekata
iz grupe od $n$ objekata.

\begin{corollary}
    $$
        \sum_{k=0}^{n} \binom{n}{k}^2 = \binom{2n}{n}
    $$
\end{corollary}

\begin{example}
    Izračunajte $\displaystyle \binom{25}{0}\binom{27}{20}+\binom{25}{1}\binom{27}{19}+\dots+\binom{25}{19}\binom{27}{1}+\binom{25}{20}\binom{27}{0}$.
\end{example}

$$
    \binom{25}{0}\binom{27}{20}+\binom{25}{1}\binom{27}{19}+\dots+\binom{25}{19}\binom{27}{1}+\binom{25}{20}\binom{27}{0} = \sum_{k=0}^{20} \binom{15}{k}\binom{27}{20-k} = \binom{27+25}{20} \approx 1.2599 \times 10^{14}
$$


\subsection{Unimodalnost}

Ako je $n$ paran prirodan broj vrijedi:
$$
\binom{n}{0}<\binom{n}{1}<\dots<\binom{n}{\frac{n}{2}},\, \binom{n}{\frac{n}{2}}>\dots>\binom{n}{n-1}>\binom{n}{n}
$$

\noindent
Ako je $n$ neparan prirodan broj vrijedi:
$$
\binom{n}{0}<\binom{n}{1}<\dots<\binom{n}{\frac{n-1}{2}} = \binom{n}{\frac{n+1}{2}}>\dots>\binom{n}{n-1}>\binom{n}{n}
$$


\begin{example}
    $2n - 1$ studenata želi osnovati svoju delegaciju od $k$ studenata. Za $k = 40$ je
    broj mogućih različitih odabira članova delegacije najveći. Odredite $n$.
\end{example}

Zbog unimodalnosti binomnog koeficijenta znamo da gornji broj mora biti jednak
$k\cdot 2 \pm 1$, jer se radi o neparnom broju studenata te je
$\dbinom{n}{\frac{n\pm 1}{2}}$ najveći binom. Dakle broj studenata je $79$.

\subsection{Zadaci}

\begin{example}
    U binomnom razvoju $\left({1\over x} + x\right)^n$ koeficijent drugog člana
    manji je od koeficijenta trećeg člana za 35. Odredite, ako postoji, član
    razvoja koji ne sadrži $x$.
\end{example}

\begin{example}
    Izračunajte vrijednost Möbiusove funkcije za sve uređene parove elemenata
    skupa $X=\{1,2,3,5,9,12\}$, ako je uređaj na skupu dan relacijom "biti
    djeljitelj od".
\end{example}
