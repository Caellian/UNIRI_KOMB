\section{Funkcije izvodnice}

Neka je $(a_n)_{n\in\mathbb{N}_0}$ niz kompleksnih brojeva.

\textbf{Obična funkcija izvodnica} (FI) niza $(a_n)_{n\in\mathbb{N}_0}$ je formalni red potencija
$$
    a(t) = a_0 + a_1t+a_2t^2 + \dots + a_nt^n+\dots = \sum_{n \geq 0} a_nt^n\,.
$$

Ako je zadana funkcija izvodnica $a(t)$, onda je

$$
a_n = \frac{a^{(n)}(0)}{n!}\,.
$$

Zatvorne formule nekih redova potencija:

\begin{itemize}
    \item $\sum_{n\geq 0} x^n = \frac{1}{1-x}$
    \item $\sum_{n\geq 0} (ax)^n = \frac{1}{1-ax}$
    \item $\sum_{n\geq 0} \binom{-n}{k}(-x)^k = \frac{1}{(1-x)^n}$
\end{itemize}

\subsection{Eksponencijalne funkcije izvodnice}
\subsection{Primjena funkcija izvodnica u kombinatorici}
