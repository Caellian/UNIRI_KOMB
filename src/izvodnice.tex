\section{Funkcije izvodnice}

Funkcija izvodnica (engl. \textit{generating function}) je formalna struktura
koja je bliska nizevima brojeva, ali dozvoljava manipulaciju niza kao jednog
cjelovitog objekta, s ciljem njegovog boljeg razumjevanja.

\begin{definition}
    Neka je $(a_n)_{n\in\mathbb{N}_0}$ niz kompleksnih brojeva.
    
    \textbf{Obična funkcija izvodnica} (FI) niza $(a_n)_{n\in\mathbb{N}_0}$ je formalni red potencija
    $$
        a(t) = a_0 + a_1t+a_2t^2 + \dots + a_nt^n+\dots = \sum_{n \geq 0} a_nt^n\,.
    $$
\end{definition}

Ako je zadana funkcija izvodnica $a(t)$, onda je

$$
a_n = \frac{a^{(n)}(0)}{n!}\,.
$$

Zatvorne formule nekih redova potencija:

\begin{itemize}
    \item $\displaystyle \sum_{n\geq 0} x^n = \frac{1}{1-x}$
    \item $\displaystyle \sum_{n\geq 0} (ax)^n = \frac{1}{1-ax}$
    \item $\displaystyle \sum_{n\geq 0} \binom{-n}{k}(-x)^k = \frac{1}{(1-x)^n}$
    \item $\displaystyle \sum_{n\geq 0} \frac{x^n}{n!} = e^x$
    \item $\displaystyle \sum_{k=0}^{n}x^k = \frac{1-x^{n+1}}{1-x}$
\end{itemize}

\subsection{Eksponencijalne funkcije izvodnice}

\begin{definition}
    Neka je $(a_n)_{n \in \mathbb{N}_0}$ niz kompleksnih brojeva.
    
    \textbf{Eksponencijalna funkcija izvodnica} (EFI) niza $(a_n)_{n \in \mathbb{N}_0}$ je formalni red potencija:
    $$
        e(t) = a_0+\frac{a_1}{1!}t + \frac{a_2}{2!}t^2 + \dots + \frac{a_n}{n!}t^n + \dots = \sum_{n\geq 0}^ \frac{a_n}{n!}t^n\,.
    $$
\end{definition}

Ako je zada funkcija izvodnica $e(t)$, onda je $a_n = e^{(n)}(0)$.

\begin{enumerate}
    \item Funkcije izvodnice $f(x) = \sum_{n\geq 0}a_nx^n$ i
    $g(x) = \sum_{n\geq 0}b_nx^n$ su jednake ako i samo ako je
    $$
        a_n = b_n\,, \forall n \in \mathbb{N}_0\,.
    $$
    \item Zbroj funkcija izvodnica $f(x) = \sum_{n\geq 0} a_nx^n$ i
    $g(x) = \sum_{n\geq 0}b_nx^n$ je red potencija
    $$
        \sum_{n\geq 0} (a_n + b_n)x^n\,.
    $$
    \item Zbroj funkcija izvodnica $f(x) = \sum_{n\geq 0} a_nx^n$ i
    $g(x) = \sum_{n\geq 0}b_nx^n$ je red potencija
    \begin{gather*}
        \sum_{n\geq 0} c_nx^n\,, \text{ gdje su:}\\
        c_0=a_0b_0\,,\quad c_1 = a_0b_1 + a_1b_0\,,\quad \dots\,,\quad c_n = \sum_{k=0}^n a_kb_{n-k}\,.
    \end{gather*}

    \item Formalna derivacija funkcije izvodnice $f(x) = \sum_{n\geq 0} a_nx^n$ je red potencija
    $$
        \sum_{n\geq 0} (n+1)a_{n+1}x^n\,.
    $$
    \item Formalni integral funckije izvodnice $f(x) = \sum_{n\geq 0} a_n x^n$ je red potencija
    $$
        \sum_{n\geq 0} \frac{1}{n} a_{n-1}x^n\,.
    $$
\end{enumerate}

\subsection{Primjena funkcija izvodnica u kombinatorici}

\begin{example}
    Primjenom funkcija iznodnica odredite broj $k$-kombinacija skupa od $n$ elemenata.
\end{example}

\begin{example}
    Primjenom funkcija izvodnica odredite broj načina na koje se mogu poredati po četiri slova iz riječi $\mathrm{MAMICA}$.
\end{example}
