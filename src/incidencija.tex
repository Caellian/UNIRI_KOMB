\textbf{Incidencijska struktura} $\mathcal{D}$ je uređena trojka
$(\mathcal{P}, \mathcal{B}, \mathcal{I})$, gdje su
$\mathcal{P}$ i $\mathcal{B}$ neprazni disjunktni skupovi i
$\mathcal{I} \subseteq \mathcal{P} \times \mathcal{B}$.
Elementi skupa $\mathcal{P}$ se nazivaju \textbf{točke}, elementi skupa
$\mathcal{B}$ \textbf{blokovi}, a relacija $\mathcal{I}$
\textbf{relacija incidencije}.

Broj blokova koji su incidentni s točkom $P$ naziva se \textbf{stupanj točke}
$P$ i broj točaka koje su incidentne s blokom $x$ naziva se
\textbf{stupanj bloka} $x$.

\bigskip
\noindent
Neka je $\mathcal{D} = (\mathcal{P}, \mathcal{B}, \mathcal{I})$ incidencijska
struktura te neka je $\mathcal{P}' \subseteq \mathcal{P}$,
$\mathcal{B}' \subseteq \mathcal{B}$,
$\mathcal{I}' = \mathcal{P}' \times \mathcal{B}'$ i
$\mathcal{I}' \subseteq \mathcal{I}$. Incidencijska struktura
$\mathcal{D}' = (\mathcal{P}', \mathcal{B}', \mathcal{I}')$ naziva se
\textbf{podstruktura} strukture $\mathcal{D}$.

\smallskip
\noindent
Neka je $\mathcal{D} = (\mathcal{P}, \mathcal{B}, \mathcal{I})$ incidencijska
struktura. Struktura $\mathcal{D} = (\mathcal{P}, \mathcal{B}, \mathcal{I}')$,
gdje je $\mathcal{I}' = \mathcal{P} \times \mathcal{B} - \mathcal{I}$ naziva se
\textbf{komplementarna struktura} strukture $\mathcal{D}$.

\smallskip
\noindent
Konačna incidencijska struktura je \textbf{kvadratna} ako je broj točaka jednak
broju blokova.

\bigskip
\noindent
Neka je $\mathcal{D} = (\mathcal{P}, \mathcal{B}, \mathcal{I})$ konačna
incidencijska struktura takva da je $|\mathcal{P}| = v$ i $|\mathcal{B}| = b$.
Označimo elemente skupa $\mathcal{P}$ sa $P_1, \dots, P_v$ i elemente skupa
$\mathcal{B}$ sa $x_1, \dots, x_b$. \textbf{Matrica incidencije} incidencijske
strukture $\mathcal{D}$ je $v \times b$ matrica $\mathbf{M} = (m_{ij})$
$$
m_{ij} = \begin{cases*}
    1, (P_i, x_j) \in \mathcal{I},\\
    0, (P_i, x_j) \notin \mathcal{I}.
\end{cases*}
$$

\noindent
Neka je $\mathbf{M}$ matrica incidencije strukture $\mathcal{D}$. Matrica
incidencije komplementarne strukture $\mathcal{D}'$ je
$\mathbf{J} - \mathbf{M}$, gdje je $\mathbf{J}$ matrica jedinica.

\bigskip
\noindent
Neka su $\mathcal{D} = (\mathcal{P}, \mathcal{B}, \mathcal{I})$ i
$\mathcal{D}' = (\mathcal{P}', \mathcal{B}', \mathcal{I}')$ incidencijske
strukture.

\noindent
Bijektivno preslikavanje
$f: \mathcal{P} \times \mathcal{B} \to \mathcal{P}' \times \mathcal{B}'$ je
\textbf{izomorfizam} iz $\mathcal{D}$ na $\mathcal{D}'$ ako vrijedi:
\begin{enumerate}
    \item $f$ preslikava $\mathcal{P}$ na $\mathcal{P}'$ i $\mathcal{B}$ na $\mathcal{B}'$
    \item $(P, x) \in \mathcal{I} \Rightarrow (f (P), f (x)) \in \mathcal{I}', \forall P \in \mathcal{P}\text{ i }\forall x \in \mathcal{B}$
\end{enumerate}

Ako je $\mathcal{D}' = \mathcal{D}$, onda se preslikavanje $f$ naziva
\textbf{automorfizam}.

\section{Primjeri incidencijskih struktura}

\begin{itemize}
    \item dizajni\newline
    Konačna incidencijska struktura
    $\mathcal{D} = (\mathcal{P}, \mathcal{B}, \mathcal{I})$ je
    $t - (v, k, \lambda)$ dizajn ako vrijedi sljedeće:
    \begin{enumerate}
        \item $|\mathcal{P}| = v$,
        \item svaki element skupa $\mathcal{B}$ incidentan je s točno $k$
        elemenata skupa $\mathcal{P}$,
        \item svakih $t$ elemenata skupa $\mathcal{P}$ incidentno je s točno
        $\lambda$ elemenata skupa $\mathcal{B}$.
    \end{enumerate}
    \item grafovi\newline
    Neka je $\mathcal{G} = (\mathcal{V}, \mathcal{E}, \mathcal{I})$ konačna
    incidencijska struktura. $\mathcal{G}$ je graf ako je svaki element skupa
    $\mathcal{E}$ incidentan s dva (ne nužno različita) elementa skupa
    $\mathcal{V}$. Elementi skupa $\mathcal{V}$ se nazivaju vrhovi i elementi
    skupa $\mathcal{E}$ bridovi.
\end{itemize}
