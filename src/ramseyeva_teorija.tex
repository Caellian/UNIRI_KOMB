Ramseyevu teorija je nazvana po Frank Plumpton Ramseyu (1903. - 1930.).

Osnovna ideja iza Ramseyeve teorije je da kada se neki veliki objekt razdijeli
na dva dijela (ili mali broj dijelova), onda barem jedan od tih dijelova mora
sadržavati vrlo lijep podskup (engl. \textit{very nice subset}).

\begin{definition}[graf]
    \textbf{Graf} je struktura koja se sastoji od \textbf{vrhova} i njihovih
    spojnica (\textbf{bridova}).
\end{definition}

\begin{definition}[potpuni graf]
    \textbf{Potpuni graf} je graf u kojemu su svaka dva vrha spojena jednim
    bridom.
\end{definition}

\begin{definition}[podgraf]
    Neka je $\mathcal{V}$ skup vrhova grafa $\mathcal{G}$ i $\mathcal{E}$ skup
    bridova grafa $\mathcal{G}$. \textbf{Podgraf} grafa $\mathcal{G}$ je graf
    čiji je skup vrhova podskupa skupa $\mathcal{V}$ i skup bridova podskup
    skupa $\mathcal{E}$.
\end{definition}

\begin{definition}[čisti podgraf]
    Neka je $\mathcal{G}$ graf čiji su bridovi obojani. \textbf{Čisti podgraf}
    je potpuni podgraf grafa $\mathcal{G}$ čiji su bridovi obojani jednom bojom.
\end{definition}

\begin{problem}
    Potpuni graf sa šest vrhova čiji su bridovi obojani dvjema bojama sadrži
    čisti podgraf s 3 vrha.
\end{problem}

Potpuni graf sa šest vrhova ima ukupno 15 bridova. Ako su oni pobojani s dvije
boje, zbog dirichletovog načela znamo da mora postojati barem jedna boja kojom
je pobojano 8 ili više vrhova.

Jer je $8 \geq 3$, tvrdnja vrijedi.

\begin{problem}
    U skupini od 6 osoba postoje ili tri osobe koje se međusovno poznaju ili tri
    osobe od kojih se nikoje dvije ne poznaju.
\end{problem}

Dokaz je identičan prethodnom.
