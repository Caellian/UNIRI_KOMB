\section{Uvod u Ramseyevu teoriju}

Nazvana je po Frank Plumpton Ramseyu (1903. - 1930.).

\begin{definition}
    \textbf{Graf} je struktura koja se sastoji od \textbf{vrhova} i njihovih
    spojnica (\textbf{bridovi}).
\end{definition}

\begin{definition}
    \textbf{Potpuni graf} je graf u kojemu su svaka dva vrha spojena jednim
    bridom.
\end{definition}

\begin{definition}
    Neka je $\mathcal{V}$ skup vrhova grafa $\mathcal{G}$ i $\mathcal{E}$ skup
    bridova grafa $\mathcal{G}$. \textbf{Podgraf} grafa $\mathcal{G}$ je graf
    čiji je skup vrhova podskupa skupa $\mathcal{V}$ i skub bridova podskup
    skupa $\mathcal{E}$.
\end{definition}

\begin{definition}
    Neka je $\mathcal{G}$ graf čiji su bridovi obojani. \textbf{Čisti podgraf}
    je potpuni podgraf grafa $\mathcal{G}$ čiji su bridovi obojani jednom bojom.
\end{definition}

\begin{theorem}
    Potpuni graf sa šest vrhova čiji su bridovi obojani dvjema bojama sadrži
    čisti podgraf s 3 vrha.
\end{theorem}

\begin{theorem}
    U skupini od 6 osoba postoje ili tri osobe koje se međusovno poznaju ili tri
    osobe od kojih se nikoje dvije ne poznaju.
\end{theorem}
