\section{Kombinacije}

Neka je $S$ skup od $n$ elemenata i neka je $r\leq n,\,r\in\mathbb{N}_0$.
Podskup skupa $S$ koji ima $r$ elemenata je \textbf{$r$-kombinacija} skupa $S$
(ili kombinacija $r$-tog razreda bez ponavljanja skupa od $n$ elemenata).

\begin{example}
    $\{1,2,7\}$ je jedna 3-kombinacija skupa $\{1,2,3,4,5,6,7\}$.
\end{example}

\subsection{K-kombinacije n-članog skupa}

$\binom{n}{r}$ je oznaka za broj $r$-kombinacija skupa od $n$ elemenata.

\begin{itemize}
    \item $\displaystyle \binom{0}{0} := 1$
    \item $\displaystyle \binom{n}{r} := 0,\quad n < r$
    \item $\displaystyle \binom{0}{r} := 1,\quad r \in \mathbb{N}$
\end{itemize}

\begin{theorem}
    Neka je $n \in \mathbb{N}, r \in \mathbb{N}_0$ i $r\leq n$. Tada je
    $$
    \binom{n}{r} = \frac{n!}{r!\cdot (n-r)!} = \frac{n\cdot (n-1)\cdot \dots \cdot (n-r+1)}{r!}.
    $$
\end{theorem}
