\section{Dirichletovo načelo}

\subsection{Slaba forma}

\begin{theorem}
    Neka je $n$ prirodan broj. Ako $n+1$ predmeta bilo kako rasporedimo u $n$
    kutija, onda barem jedna kutija sadrži barem dva predmeta.
\end{theorem}

\begin{example}
    Neka su $S$ i $T$ konačni skupovi takvi da je $|S| > |T|$ i neka je $f:S\to
    T$ funkcija. Može li $f$ biti injektivna funkcija?
\end{example}

Funkcija $f$ mora biti surjektivna jer vrijedi $|S| > |T|$. Kako bi funkcija
bila injektivna, mora vrijediti $|S| \leq |T|$.

\begin{example}
    Na otvaranje nove izložbe poznatog slikara okupilo se društvo od 200 osoba.
    Dokažite da među prisutnima ima barem dvije osobe s istim brojem poznanika
    na izložbi.
\end{example}

\begin{example}
    \begin{itemize}
        \item Ako $2n+1$ predmet raspoređujemo u $n$ kutija, tada barem jednak
        broj kutija sadrži barem $\underline{\qquad}$ predmeta.
        \item Ako $3n+1$ predmet raspoređujemo u $n$ kutija, tada barem jednak
        broj kutija sadrži barem $\underline{\qquad}$ predmeta.
        \item Ako $n(k-1)+1$ predmet raspoređujemo u $n$ kutija, tada barem
        jednak broj kutija sadrži barem $\underline{\qquad}$ predmeta.
    \end{itemize}
\end{example}

\subsection{Jaka forma}

\begin{theorem}
    Neka su $n$ i $m$ prirodni brojevi. Ako je $m$ predmeta raspoređeno u $n$
    kutija, onda barem jedna kutija sadrži barem $\lfloor \frac{m-1}{n} \rfloor
    + 1$ predmeta.
\end{theorem}

\begin{example}
    Pismenom ispitu iz Kombinatorike pristupilo je 30 studenata. Rješavajući
    zadatke niti jedan student nije napravio više od šest grešaka. Dokažite da
    je barem pet studenata imao jednak broj grešaka na tom ispitu.
\end{example}

\subsection{Opća forma}

\begin{theorem}
    Neka su $n, r_1, \dots, r_n$ prirodni brojevi. Ako je
    $r_1+r_2+\dots+r_n-n+1$ predmeta raspoređeno u $n$ kutija $K_1, K_2, \dots,
    K_n$, onda barem jedna kutija $K_i$ sadrži barem $r_i$ predmeta.
\end{theorem}

\begin{example}
    Na raspolaganju imamo 15 crvenih balona, 20 bijelih balona, 10 plavih balona
    i 30 žutih balona. 72 od tih balona trebamo rasporediti u četiri različite
    kutije: crvenu, bijelu, plavu i žutu kutiju. Dokažite da
    \begin{itemize}
        \item ili crvena kutija sadrži barem 15 balona
        \item ili bijela kutija sadrži barem 20 balona
        \item ili plava kutija sadrži barem 10 balona
        \item ili žuta kutija sadrži barem 30 balona.
    \end{itemize}
\end{example}

\subsection{Princip prentinca za golube}

Neka su $m$ i $n$ prirodni brojevi. Neka je $m$ golubova smješteno u $n$
golubnjaka. Tada vrijedi:

\begin{itemize}
    \item ako je $m>n$, onda postoji golubnjak u kojemu se nalaze barem dva
    goluba.
    \item Ako je $m=n$, onda postoji prazan golubnjak ako i samo ako postoji
    golubnjak u kojemu su barem dva goluba.
    \item Ako je $m<n$, onda je barem jedan golubnjak prazan.
\end{itemize}

