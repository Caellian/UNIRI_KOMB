\section{Permutacije skupova}

Neka je $S$ skup od $n$ elemenata i neka je $r\leq n$ prirodan broj. Uređena
$r$-torka $(x_1,x_2,\dots,x_r)$, gdje su $x_1,\dots,x_r$ međusobno različiti
elementi skupa $S$, je \textbf{$r$-permutacija} skupa $S$ (ili varijacija bez
ponavljanja $r$-tog razreda skupa od $n$ elemenata).

\begin{example}
    Uređena petorka $(7,1,5,3,4)$ je jedna 5-permutacija skupa $S=\{1,2,3,4,5,6,7\}$.
\end{example}

\begin{theorem}
    Broj injektivnih funkcija skupa od $r$ elemenata u skup od $n$ elemenata
    jednak je broju $r$-permutacija skupa od $n$ elemenata.
\end{theorem}

Broj $r$-permutacija skupa veličine $n$ je:

$$
P(n, r) =
    \begin{Bmatrix}
        0, & n < r\\
        1, & r = 0\\
        n \times (n - 1) \times (n - 2) \times \dots \times (n - r + 1), & \text{inače}
    \end{Bmatrix}
    = \frac{n!}{(n-r)!}
$$

Neka je $S$ skup od $n$ elemenata. \textbf{Permutacija} skupa $S$ je
$n$-permutacija.

Skup od $n$ elemenata se može posložiti na $n!$ različitih načina, tj. u $n!$
različitih permutacija. Vrijedi:

\begin{gather*}
    P_n = P(n, n) = n \cdot (n-1) \cdot \dots \cdot 2 \cdot 1 = n!\\
    0! := 1
\end{gather*}

\subsection{Leksikografski poredak}

Leksikografski poredak je jednostavan i fleksibilan način za generiranje svih
permutacija zadanog skupa koji podržava i ponavljajuće vrijednosti.

Funkcionira tako da se svim elementima skupa dodijele težinske vrijednosti te ih
se redom ispisuje kao u riječniku (iz čega dobiva naziv).

Narayana Pandita je u 14. st. napisao jedan od prvih poznatih algoritama za
generiranje leksikografskog poretka:

\begin{itemize}
    \item Pronađi najveći indeks \verb|k| takav da \verb|a[k] < a[k+1]|.
    \begin{itemize}
        \item Ako takav indeks ne postoji, radi se o zadnjoj permutaciji.
    \end{itemize}
    \item Pronađi najveći indeks \verb|l| veći od \verb|k| takav da \verb|a[k] < a[l]|.
    \item Zamjeni vrijendosti \verb|a[k]| i \verb|a[l]|.
    \item Obrni redoslijed elemenata od \verb|a[k+1]| do (i uključujući) zadnji
    element \verb|a[n]|.
\end{itemize}

Ova metoda koristi približno 3 usporedbe i 1.5 zamjene po permutaciji,
amortizirano za cijeli skup, ne računajući inicijalno sortiranje.

\noindent
Vremenska složenost određivanja leksikografskog poreta je $\mathcal{O}(n\cdot n!)$.


\begin{problem}
    Koji je leksikografski poredak skupa $\{A, B, \symbol{"2660}, B\}$?
\end{problem}

\begin{multicols}{4}
\raggedright
\noindent
1. $(A, B, \symbol{"2660}, B)$,\linebreak
2. $(A, B, B, \symbol{"2660})$,\linebreak
3. $(A, \symbol{"2660}, B, B)$,\linebreak
4. $(A, \symbol{"2660}, B, B)$,\linebreak
5. $(A, B, B, \symbol{"2660})$,\linebreak
6. $(A, B, \symbol{"2660}, B)$,\linebreak

\columnbreak

\noindent
7.\quad $(B, A, \symbol{"2660}, B)$,\linebreak
8.\quad $(B, A, B, \symbol{"2660})$,\linebreak
9.\quad $(B, \symbol{"2660}, A, B)$,\linebreak
10. $(B, \symbol{"2660}, B, A)$,\linebreak
11. $(B, B, A, \symbol{"2660})$,\linebreak
12. $(B, B, \symbol{"2660}, A)$,

\columnbreak

\noindent
13. $(\symbol{"2660}, A, B, B)$,\linebreak
14. $(\symbol{"2660}, A, B, B)$,\linebreak
15. $(\symbol{"2660}, B, A, B)$,\linebreak
16. $(\symbol{"2660}, B, B, A)$,\linebreak
17. $(\symbol{"2660}, B, A, B)$,\linebreak
18. $(\symbol{"2660}, B, B, A)$,

\columnbreak

\noindent
19. $(B, A, B, \symbol{"2660})$,\linebreak
20. $(B, A, \symbol{"2660}, B)$,\linebreak
21. $(B, B, A, \symbol{"2660})$,\linebreak
22. $(B, B, \symbol{"2660}, A)$,\linebreak
23. $(B, \symbol{"2660}, A, B)$,\linebreak
24. $(B, \symbol{"2660}, B, A)$
\end{multicols}

\begin{problem}
    Koja je po redu u leksikografskom poretku $(2, 3, 1, 4)$ permutacija za skup $\{1, 2, 3, 4\}$?
\end{problem}

Preskačemo permutacije koje počinju s:
\begin{itemize}
    \item $1$: $3!$,
    \item $21$: $3! + 2!$
\end{itemize}

Dakle permutacija $(2, 3, 1, 4)$ je 8. po redu u leksikografskom poretku.

\begin{problem}
    Koja je prethodna, a koja sljedeća permutacija permutaciji $(2, 3, 1, 4)$ iz prethodnog zadatka?
\end{problem}

Prethodna permutacija je $(2, 1, 4, 3)$.
Sljedeća permutacija je $(2, 3, 4, 1)$.

\subsection{Ciklički tip permutacije}

Neka je $k_i$ broj ciklusa duljine $i$ (duljina ciklusa je broj elemenata u ciklusu) permutacije $\pi : \mathbb{N}_n \to \mathbb{N}_n$. Tada kažemo da je permutacija $\pi$ cikličkog tipa $(k_1, k_2, \dots, k_n)$.

\begin{theorem}[broj permutacija cikličkog tipa]
    Broj permutacija skupa od $n$ elemenata cikličkog tipa $(k_1, k_2, \dots,
    k_n)$ je jednak:

    $$
        \frac{n!}{1^{k_1}\cdot k_1! \cdot 2^{k_2}\cdot k_2! \cdot \dots \cdot n^{k_n} \cdot k_n!}
    $$
\end{theorem}

\begin{problem}
    Koji je ciklički zapis sljedećih permutacija:
    \begin{enumerate}
        \item 415362
        \item 123465
    \end{enumerate}
\end{problem}

\begin{enumerate}
\item $\displaystyle
\begin{pmatrix}
    1&2&3&4&5&6\\
    4&1&5&3&6&2
\end{pmatrix} = (143562)
$

\item $\displaystyle
\begin{pmatrix}
    1&2&3&4&5&6\\
    1&2&3&4&6&5
\end{pmatrix} = \underbrace{(1)(2)(3)(4)}_{\substack{\mkern-15mu \text{cikluse duljine }1 \mkern-15mu\\ \mkern-20mu \text{ nije potrebno pisati} \mkern-20mu}}(56) = (56)
$
\end{enumerate}