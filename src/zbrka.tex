\section{Totalna zbrka}

\textbf{Totalna zbrka} ili dearanžman je parmutacija elemenata skupa u kojoj se
niti jedan element ne pojavljuje u svojoj originalnoj poziciji, tj. nema fiksnih
točaka.

\bigskip
\noindent
$D_n$ je broj dearanžmana ili totalne zbrke nekog skupa. Također se naziva i Montmortov broj. Druge oznake su $!n$, $n\text{!`}$.

\begin{theorem}
    Za prirodan broj $n$ vrijedi:
    $$
        D_n = \begin{Bmatrix}
            (n-1) \cdot (D_{n - 1} + D_{n - 2}),&n \geq 2\\
            0,&n = 1\\
            1,&n = 0
        \end{Bmatrix} = n! \left(1-\frac{1}{1!} + \frac{1}{2!} - \frac{1}{3!} + \dots + (-1)^n \frac{1}{n!}\right).
    $$
\end{theorem}

\textbf{Fiksna točka} je točka čija se pozicija ne mijenja prilikom preslikavanja.

\begin{example}
    Neka $f:S\to T$. Element $x\in S$ je fiksna točka funkcije ako vrijedi $f(x)
    = x$.

    \begin{enumerate}
        \item Koliko ima permutacija skupa $\mathbb{N}_n$ takvih da su sve točke
        skupa $\mathbb{N}_n$ fiksne točke te permutacije?
        \item Koliko ima permutacija skupa $\mathbb{N}_n$ takvih da je 1 filsna
        točka te permutacije?
        \item Koliko ima permutacija skupa $\mathbb{N}_n$ bez fiksnih točaka?
        \begin{itemize}
            \item $D_0 \coloneq 1$
            \item $D_1 = ?$
            \item $D_2 = ?$
        \end{itemize}
    \end{enumerate}
\end{example}

\begin{enumerate}
    \item Postoji jedna takva permutacija, i u njoj je redoslijed elemenata
    jednak originalnom skupu $S$. U tom slučaju vrijedi $f \equiv \mathrm{I} : S
    \to S$.
    \item Postoji $n \cdot (n - 1)!$ takvih permutacija. Možemo odabrati bilo
    koji od $n$ elemenata da bude fiksna točka, a preostalih $n-1$ elemenata
    možemo permutirati na $(n-1)!$ načina.
    \item Broj permutacija bez fiksnih točaka je jednak $D_n$ po gornjoj formuli.
    \begin{itemize}
        \item $D_0 \coloneq 1$
        \item $D_1 = 0$
        \item $D_2 = 1$ (zamjena dva elementa)
    \end{itemize}
\end{enumerate}
