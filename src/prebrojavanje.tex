\section{Načela prebrojavanja}

\subsection{Načelo zbroja (ili)}
$S_1, \dots, S_n$, $n \in \mathbb{N}$, $S$ su konačni disjunktni skupovi ($S_i
\cap S_j = \emptyset, i\neq j$)
\begin{align}
    \left|\bigcup_{i=1}^nS_i\right| &= \sum_{i=1}^{n}|S_i| \\
    |S_1 \cup S_2 \cup \dots \cup S_n| &= |S_1| + |S_2| + \dots + |S_n|\nonumber
\end{align}

\begin{example}
    Na koliko načina možemo između 5 muškaraca, 6 žena i 7 djece odabrati
    jednu osobu?
\end{example}

\textit{Odgovor:} $5+6+7 = 18$

\begin{example}
    Iz jednog grada vodi 5 cesti na sjever, 3 ceste na jug, 4 ceste na
    zapad i 6 cesti na istok. Na koliko načina možemo izaći iz grada?
\end{example}

\textit{Odgovor:} $5+3+4+6 = 18$

\subsection{Načelo produkta (i)}
$S_1, \dots, S_n$, $n \in \mathbb{N}$, $S$ su konačni disjunktni skupovi
\begin{align}
    |S_1 \times S_2 \times \dots \times S_n| &= \prod_{i=1}^{n}|S_i| \\
    &= |S_1| \cdot |S_2| \cdot \ldots \cdot |S_n|\nonumber
\end{align}

\begin{example}
    Na koliko načina možemo između 5 muškaraca, 6 žena i 7 djece odabrati jednog
    muškarca, jednu ženu i jedno dijete.
\end{example}

\textit{Odgovor:} $5\cdot 6\cdot 7 = 210$

\begin{example}
    Neka je $S$ skup koji ima $n$ elemenata i $T$ skip koji ima $m$ elemenata.
    Odredite broj funkcija sa skupa $S$ u skup $T$.
\end{example}

\subsection{Načelo jednakosti}
$$
    |S| = |T| \iff \exists \text{ bijekcija } f: s \to t
$$

\subsection{Zadaci}

\begin{example}
    Koliko ima 6-znamenkastih brojeva koji se sastoje od 3 znamenke 1 i 3
    znamenke 2?
\end{example}

\begin{example}
    Koliko ima 6-znamenkastih brojeva koji se sastoje od 3 znamenke 1 i 3
    znamenke 2 ako se ne smije početi s 0?
\end{example}

\begin{example}
    Smještamo 32 studenta i 12 studentica u 2 učionice sa 30 mjesta,
    \begin{enumerate}
        \item NKN ih možemo smjestiti u učionice,
        \item NKN ih možemo smjestiti u učionice ako u svakoj učionici mora biti
        minimalno 6 studentica?
    \end{enumerate}
\end{example}

\begin{example}
    Na koliko načina se 7 majica može raporediti u 18 ladica ako:
    \begin{enumerate}
        \item su sve majice različite?
        \item su sve majice različite i ako u svakoj ladici smije biti najviše jedna majica?
        \item su sve majice jednake i ako u svakoj ladici smije biti najviše jedna majica?
    \end{enumerate}
\end{example}
